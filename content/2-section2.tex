\cleardoublepage
% \newpage

% % 页眉:单页印刷
% \fancyhead[L]{\zihao{-5}{\songti 重庆大学本科学生毕业论文(设计)}}
% \fancyhead[R]{\zihao{-5}{\songti 2\quad 正文文字格式}}

% 页眉:双页印刷
\fancyhead[CE]{\zihao{-5}{\songti 重庆大学本科学生毕业论文(设计)}}
\fancyhead[CO]{\zihao{-5}{\songti 2\quad 正文文字格式}}

\section{正文文字格式}
\subsection{论文正文}
论文正文是主体,一般由标题、文字叙述、图、表格和公式等部分构成。一般可包括理论分析、计算方法、实验装置和测试方法,经过整理加工的实验结果分析和讨论,与理论计算结果的比较以及本研究方法与已有研究方法的比较等,因学科性质不同可有所变化。\par

\subsection{字数要求}
\subsubsection{本科论文字数要求}
论文主体部分字数要求:理工类专业一般不少于1.5万字,其他专业一般不少于1.0万字。

\subsection{本章小结}
本章介绍了……