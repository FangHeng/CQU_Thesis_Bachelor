\fancyhead [RO, L] {\zihao{-5}{\songti 1\quad 绪论}}
\fancyhead [LO, R] {\zihao{-5}{\songti 重庆大学本科学生毕业论文(设计)}}




\newpage
\fancyhead[LH]{\zihao{-5}{\songti 重庆大学本科学生毕业论文(设计)}}
\fancyhead[RH]{\zihao{-5}{\songti 1\quad 绪论}}
\pagenumbering{arabic}


\titleformat{\section}{\heiti\zihao{3}\centering}{\thesection}{0.5em}{}[]
\section{绪论}
\subsection{引言}
\zihao{-4} 
学位论文的引言是研究工作的开端,旨在引导读者对研究的背景、问题、目的和重要性有所了解。在背景介绍中,阐述与研究主题相关的一般性信息,为引入具体问题做铺垫。问题陈述明确阐述研究的核心问题,强调研究的重要性。进一步,目的和目标部分解释研究的意图和预期贡献。对研究范围和限制进行概述,确立研究的边界和可能的局限性。在方法论概述中简要介绍选择的研究方法及其与问题的关联。最后,通过论文结构概述,为读者提供整篇论文的框架,使其能够预知各章节内容。整个引言力求简明扼要,引发读者兴趣,并确保与后续章节之间的连贯性,以保持整体逻辑清晰……

\subsection{本文主要研究内容}
\zihao{-4} 
研究内容部分应着重介绍你具体研究的对象、问题或主题。明确概括你的研究范围和涉及的关键要素。强调你的研究如何填补领域现有知识的空白或者为该领域做出独特的贡献。这一部分的目标是为读者提供关于研究的具体方向和深度的清晰认识……

\subsection{本文研究意义}
\zihao{-4} 
在研究意义部分,阐述你的研究对学术界和实践领域的重要性。强调你的研究如何推动学科的发展,解决实际问题或者对社会产生积极影响。可以涉及到研究的创新性、实用性、理论意义等方面。此部分需要突出你的研究对于学术和实际领域的价值,使读者明白为何该研究是有意义的……

\subsection{本章小结}
\zihao{-4} 
小结部分是引言的收尾,总结引言中提到的关键信息。简要回顾研究的背景、问题、目的和结构,并为读者提供对整个研究的整体印象。强调研究的重要性,为后续章节的阅读做铺垫。小结要紧凑而有力,概括引言的核心内容,使读者对整篇论文的主旨有明确的认识……

