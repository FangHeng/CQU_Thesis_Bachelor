\newpage
\pagestyle{fancy}
\pagenumbering{Roman}

% 设置左侧页眉
\fancyhead[LH]{ \songti\zihao{-5} 重庆大学本科学生毕业论文(设计)}
\fancyhead[RH]{\songti\zihao{-5} 摘要}



\addcontentsline{toc}{section}{摘要}

\section*{摘\quad 要}
%摘要:二字间空两格,黑体三号居中,段前,段后各空一行。

摘要是论文(设计)内容不加注释和评论的简短陈述,应具有独立性和自明性,即不阅读论文(设计)的全文,就可以获得必要的信息。\par 
摘要一般应说明研究工作的目的 和意义、研究思想和方法、研究过程、研究结果和最终结论等。摘要中一般不用图、表、化学结构式、计算机程序,不用非公知公用的符号、术语和非法定的计量单位。\par 
中文摘要一般为300-500汉字。\par 
摘要页置于英文题名页后。 \par 
关键词是从论文(设计)题名、摘要或正文中选取的对表示论文(设计)主题内容起关键作用,且具有检索意义的词或词组。一般每篇论文(设计)应选取3-5个词作为关键词,以显著的字符另起一行,排在同种语言摘要的下方,尽量用《汉语主题词表》或各专业主题词表提供的规范词。\par 
关键词与摘要的内容之间空一行。关键词的词间用分号间隔,末尾不加标点。\\
~\\
\hspace*{2em}{\heiti \zihao{-4}关键词}:学位论文;论文格式;规范化;模板\\
%关键字:宋体12磅,行距20磅,段前段后0磅,关键字之间用分号隔开,关键词三个字加粗。
